\documentclass{JAC2003}
\usepackage{graphicx}
\usepackage{booktabs}
\usepackage{url}

%%   VARIABLE HEIGHT FOR THE TITLE BOX (default 35mm)
\setlength{\titleblockheight}{48mm}

\title{%
    FREE AND OPEN SOURCE SOFTWARE AT CERN:\\
    INTEGRATION OF DRIVERS IN THE LINUX KERNEL}
\author{%
	Juan David Gonz\'alez Cobas Samuel Iglesias Gons\'alvez,\\
	Julian Lewis, Javier Serrano, Manohar Vanga (CERN, Geneva),\\ 
	Emilio G. Cota (Columbia University, NY; formerly CERN), \\
	Alessandro Rubini, Federico Vaga (University of Pavia)}

\begin{document}

\maketitle
\begin{abstract}
    Most device drivers written for accelerator control systems suffer from a
    severe lack of portability due to the \emph{ad hoc} nature of the code,
    often embodied with intimate knowledge of the particular machine it is
    deployed in.
    In this paper we challenge this practice by arguing for the opposite
    approach: development \emph{in the open}, which in our case translates
    into the integration of our code within the linux kernel. We make our
    case by describing the upstream merge effort of the \texttt{tsi148}
    driver, a critical (and complex) component of the control system.
    The encouraging results from this effort have then led us to follow the
    same approach with two more ambitious projects, currently in the works:
    linux support for the upcoming FMC boards~\cite{fpga-fmc,vita-fmc}
    and a new I/O subsystem.
\end{abstract}


\section{TSI148 DRIVER INTEGRATION IN THE KERNEL}
\subsection{Rationale}
The VME bus is a central component of the Controls System at CERN. We rely
on 1140 FECs (Front End Computers), 710 of which are VME crates with SBC
(Single Board Computers). A process of renovation is in course, involving
the migration from
\begin{Itemize}
\item CES RIO2/RIO3 SBCs with PowerPC CPUs runing LynxOS (around 605
    crates by August 2011), to
\item MEN-A20 SBCs with Intel CPUs running  a real-time enabled Linux
    (around 105 by August 2011).
\end{Itemize}

The MEN-A20 SBC incorporates a TSI148 VME-to-PCI-X bridge chip.  To support
the functionalities required, and for maximum backward compatibility with
the legacy CES API, a device driver was developed at CERN group BE/CO in
spring 2009. Clearly, this is a strategic component of the controls system:
every VME device relies on the provided programming interface to the VME
bus. The CERN-developed driver offers, therefore, a CES compatibility API
to facilitate the transition during the renovation process, and a new API
more similar to common practice in the Linux kernel.

After a period of validation and improvement taking place during 2010,
the strategic decision was taken to submit the driver for inclusion in the
mainline Linux kernel tree. The ongoing process, started beginning 2011 and
in course of completion, is described in the following sections.

\subsection{Benefits}

There are many reasons that make the insertion of code in the mainline kernel
a desirable target.

\begin{Itemize}
\item Smoother maintenance in the (rather frequent) event of kernel API
    modification.
\item Very strict process of peer review of the code by knowledgeable
    and specialized maintainers.
\item Widespread distribution of the code base, which can then be
    enhanced and get contributions by researchers working on similar
    problems.
\item Input from the topmost experts in the field.
\item Best practice and use of bleeding-edge tools selected by
    experienced programmers
\item Avoidance of suboptimal, ad hoc solutions in favor of the
    best ones from the technical point of view.
\item Faster cycle of bug fixes.
\item Contributing back in return to the many benefits the FOSS community
    gives us.
\item Being able to drive a critical hardware component with software
    in the vanilla kernel, with no local, idiosyncratic modifications.
\end{Itemize}

It is an interesting fact that the latter, which in the beginning looked
a far-fetched reason for undertaking this project, turned out to be
more relevant as it evolved. Renovated FECs at CERN controls system
are diskless machines, booting a custom kernel carefully configured and
patched to match local needs. Although this kernel is essentially a vanilla
kernel from \url{http://kernel.org} with a very reduced configuration, the
\verb|CONFIG_PREEMPT_RT| patch set applied, plus many local tweaks, maintaining
this system soon becomes a full time job requiring great expertise and care,
given its radical impact in the controls system stability. Relying on a
stock kernel from a main distribution alleviates this dependency on a single,
dedicated maintainer, and pushes strongly towards having the \texttt{tsi148}
driver integrated upstream.

\subsection{Drawbacks}

Some of the drawbacks herewith enumerated will be elaborated further in
the description of the integration process, but we summarize here the most
important ones.

\begin{Itemize}
\item It's tough. One should be ready for the peculiar culture of the
    Linux Kernel Mailing List.
\item Design, APIs and coding practice that is customary or simply
    acceptable locally must be adapted or rebuilt to comply with the strict
    standards of the Linux kernel developers. The morale is that the end
    product will almost certainly be different from what one initially would
    settle for (but will also certainly be better).
\item One must be prepared to compromise. The most practical,
    short-lived solution might not be the technically perfect one kernel
    maintainers aim at.
\item Maintainers are occasionally hard to deal with.
\item The process may be long.
\item Gradual changes are likely to be better accepted that big chunks
    of code submitted out of the blue.
\item Having a history of prior contributions gives more respectability
    and ease of acceptance: the system is based on meritocracy.
\end{Itemize}

\subsection{The process}

Submission of source code for acceptance in the kernel is done by means of
patches subject to a process of peer review before acceptance. The reviews
are sometimes daunting, and much editing and re-submissions can result. This
process of strict code review is another good practice our team adopted for
its development process, a practice that has proved very beneficial.

Our fist attempt of integration of the \verb|tsi148| driver was initiated
in mid-2010 by one of the authors (Emilio Garc\'ia Cota).

By that time, a \verb|tsi148| driver had recently been admitted in
the \texttt{staging/} area of the kernel, its maintainer being Martyn
Welch. This driver brought a tentative support for the VME bus that covered
two VME-to-PCI bridges:
\begin{Itemize}
\item The Tundra Universe chips, with work derived from VMELinux by John
Huggins and Michael Wyrick.
\item The Tundra TSI148 chip, inspired in the above.
\end{Itemize}

The set of patches initially submitted provided improvements in several areas,
esp. in orthodox implementation of the Linux bus/device model concepts.
Acceptance by the maintainer was partial, and modifications related to the
device model implementation remained controversial and not accepted.

The submission was re-approached beginning 2011 by another of the authors
(Manohar Vanga). In this case, some bugs in the staging driver were fixed,
and the bulk of modifications concerning the device model were partially
acknowledged. This led to a third iteration of the process, and the
re-issue of a third patch set in August 2011. At the time of this writing,
the definitive submission and acceptance of the last set of patches (five
change sets) concerning the core device model is in its final stage, and will
be applied by the main kernel maintainer in the next merge window.

Although the whole set of patches concerning bug fixes and device model of the
\verb|tsi148| VME bridge driver is finally accepted, there is still a long
way before reaching the final desideratum, i.e., driving our \verb|tsi148|
devices with stock software from the mainline kernel tree.
\begin{Itemize}
\item The outstanding goal should be now getting the VME driver out of the
\verb|./staging/| tree. For that purpose, code must be worked on (or it'll
die), and set up to the highest standard of quality to be accepted upstream.
\item There are API  incompatibilities with the driver currently used at
CERN; this implies that a transient kernel module must adapt interfaces
if driver code has to remain untouched.
\end{Itemize}

\section{DRIVERS FOR THE FMC FAMILY}

\begin{figure}[t]
   \centering
   \includegraphics*[width=80mm]{wishbone-enum.eps}
   \caption{Wishbone bus driver architecture}
   \label{wishbone-enum}
\end{figure}

A family of carrier/mezzanine modules compliant with the ANSI VITA~57
FMC standard~\cite{vita-fmc} is described in~\cite{fpga-fmc}. This kit
of boards, developed by the BE/CO Hardware and Timing section at CERN is
another case in point for the benefits of a Linux kernel-centric
approach. The hardware concept
and architecture are described in the aforementioned paper~\cite{fpga-fmc}; the
key point is that the mezzanine provides basic circuitry, and the core of the
application logic is implemented in the FPGA of the carrier board. The cores
inside this FPGA are interconnected via a Wishbone~\cite{wishbone-spec} bus.


From the software point of view, a particular instance of this family
behaves like a PCI-to-Wishbone or VME-to-Wishbone bridge. The Wishbone
bus interconnects a set of cores providing functionalities that are either
common to the whole family of mezzanines, or specific to the application
board actually plugged in the FMC slot. We show in figure~\ref{slow-adc}
the block diagram of the FMC 100MS 14 bit ADC, a typical example of an FPGA
application comprising cores for, among others

\begin{Itemize}
\item Basic I${}^2$C interfacing to the mezzanine board.
\item Wishbone mastering.
\item DMA access to DDR3 memory in the carrier board.
\item Mezzanine-specific control logic (e.g., ADC programming/setup).
\item Interrupt control.
\end{Itemize}
The consequence of this modular design of the application FPGA is that
the device driver architecture reflects the structure of this set of cores
(see figure~\ref{wishbone-enum}). The driver for the carrier board performs
essentially the following basic functions
\begin{Itemize}
\item Identify the carrier board and initialize it.
\item Perform a basic identification of the mezzanine(s) installed in
    the FMC slot(s), and their configured applications.
\item Load the application firmware into the carrier FPGA.
\item Register a Wishbone bus with the kernel.
\item Enumerate the cores in that firmware (to wit, the
    inner blocks in figure~\ref{slow-adc}).
\item Register the devices those cores implement and install the drivers
    associated to them.
\end{Itemize}
The process is directly implied by how the Linux kernel device model
is implemented as described in~\cite{device-model} or in chapter~14
of~\cite{rubini}. Conceptually, it amounts to the provision of a bus driver
for Wishbone, plus a PCI-to-Wishbone bridge driver for the carrier board.

\begin{figure}[t]
   \centering
   \includegraphics*[width=80mm]{block_diagram.eps}
   \caption{Block diagram of the FMC slow ADC application}
   \label{slow-adc}
\end{figure}

Modularity and reusability of cores and drivers are not the only rationale
behind this design. Actually, the drivers for the FMC family are the second
family of drivers planned to be integrated in the mainline kernel. Given
that boards and their applications will be manufactured by external
companies and available to the general public and not only to CERN,
having their drivers and the Wishbone
bus integrated upstream is the logical step to take. The Wishbone enumeration is made
possible through the definition of an FPGA configuration space developed
in~\cite{fpga-config-space}, that will be the basis for the integration of
all drivers of this family in the kernel.

\section{DATA ACQUISITION DRIVERS: KERNEL FRAMEWORKS}

The third family of drivers considered for integration upstream is less
homogeneous than the FMC family. The BE/CO group supports a standard kit
of hardware modules for data acquisition and general analog and digital I/O,
some of whose characteristics can be compared in table~\ref{zio-modules}. Linux
drivers for these modules were developed in different stages of the renovation
process, leading from legacy LynxOS drivers to Linux; therefore, there is a
varying degree of adherence to standard practice among the Linux kernel
developers,
because of the LynxOS coding practice formerly prevalent.

\begin{table*}[ht]
   \centering
   \caption{BE/CO data acquisition modules}
   \begin{tabular}{llllll}
       \toprule
	\textbf{Module}& \textbf{Type}& \textbf{Channels}&
	\textbf{Resolution}& \textbf{Speed}& \textbf{Bus} \\
       \midrule
	VMOD-12E8/16    &  Analog input  & 8/16ch & 12b    & 15us/sample & VME/PCI  \\
	VD80            &  Analog input  & 16ch   & 16b    & 200kS/s     & VME  \\
	SIS3300         &  Analog input  & 8ch    & 12/14b & 100MS/s     & VME  \\
	SIS3302         &  Analog input  & 8ch    & 16b    & 100MS/s     & VME  \\
	SIS3320         &  Analog input  & 8ch    & 12b    & 250MS/s     & VME  \\
	``Fast'' FMC ADC&  Analog input  & 4ch    & 14b    & 100Ms/s     & VME/PCIe (Wishbone)  \\
	``Slow'' FMC ADC&  Analog input  & 8ch    & 16b    & 100kS/s     & VME/PCIe (Wishbone)  \\
       \midrule
	CVORB V4        &  Analog output & 16ch   &  16b   &  5us/sample & VME 	 \\
	VMOD-12A2/4     &  Analog output & 2ch    &  12b   &  10us/sample& VME/PCI \\
	CVORG           &  Analog output & 2ch    &  14b   &  100 MS/s   & VME  \\
       \midrule
	VMOD-TTL        &  Digital I/O   & 20ch   & 1b     & n/a         & VME/PCI \\
	CVORA           &  Digital I/O   & 32ch   & 1--32b & 100Mhz	 & VME \\
       \bottomrule
   \end{tabular}
   \label{zio-modules}
\end{table*}

Two needs arise here clearly, in the light of our intention of having our
whole set of drivers incorporated upstream:
\begin{Itemize}
\item Proceed to a more homogeneous, Linux-only code base.
\item Provide a general framework for data acquisition and control
    devices.
\end{Itemize}
Concerning the latter, there are two candidate Linux kernel frameworks:
Comedi~\cite{comedi} and IIO~\cite{iio}. Both are in \texttt{./staging/}, and
careful analyses show that they prove insufficient for our needs. Therefore,
the development of a more complete and acceptable framework for industrial data
acquisition and analog/digital I/O becomes part of our effort of integrating
our supported device drivers in the Linux kernel. This effort is motivated
both by the requirements of legacy drivers and by the requirements that
the FMC family will impose in driver design, esp. in the area of operating
system interface.

Such a framework, codenamed \texttt{zio}, intends to cover all the relevant
aspects of data acquisition devices, far beyond the existing frameworks,
such as
\begin{Itemize}
\item Digital and analog input and output.
\item One-shot and streaming (buffered) data acquisition or waveform play.
\item Resolution.
\item Sampling rate.
\item Buffer management and timing for streaming conversion.
\item Support for DMA.
\item Calibration, offset and gain.
\item Bit grouping in digital I/O.
\item Timestamping.
\item Triggering of acquisition/output.
\item Clean design conforming to Linux kernel practice.
\end{Itemize}
At the time of this writing, prototype versions of the framework software
are under development phase~\cite{zio-git}.

\section{FORTHCOMING STRATEGY}

The strategy for inclusion of our driver kit in the kernel relies now on a
handful of key milestones
\begin{Enumerate}
\item Initiate upstream integration of the Wishbone bus driver.
	\label{wb}
\item Make the in-tree and our local API for \verb|tsi148| converge.
	\label{api}
\item Get the VME driver out of \texttt{staging}.
\item Develop the \texttt{zio} framework as a competitive alternative
	to Comedi and IIO.
	\label{zio}
\item Initiate integration of the \texttt{sis33xx} drivers into the
    \texttt{zio} framework and the mainstream kernel.
\end{Enumerate}
Points~\ref{wb}, \ref{api} and~\ref{zio} are short-term priorities that
should come out as quite straightforward.

\section{CONCLUSIONS AND LESSONS LEARNED}

The main lessons we got from this process, that we will have to take into
account in the course of the forthcoming integrations, are
\begin{Itemize}
\item It's better to hit first. However technically sound a proposal might
be, it is better to get it accepted when it is new, and not contending with
other more entrenched positions.
\item Getting input and enhancements from Linux peer developers
is tremendously beneficial to improve the quality of the code, of the
design and of the working environment (tools, policies and style being
naturally given by what the Linux kernel developers have already tried
and tested over many years).
\end{Itemize}

\begin{thebibliography}{9}   % Use for  1-9  references
%\begin{thebibliography}{99} % Use for 10-99 references

\bibitem{fpga-fmc}
P. Alvarez, M. Cattin, J. H. Lewis, J. Serrano, T. Wlostowski,
``FPGA Mezzanine Cards for CERN’s Accelerator Control System'',
ICALEPCS'09, Kobe, October 2009, WEB002, p.~376,
\url{http://www.JACoW.org}.

\bibitem{vita-fmc}
VME International Trade Association,
``FPGA Mezzanine Card (FMC) Standard'', \url{http://www.vita.com/}

\bibitem{wishbone-spec}
OpenCORES,
``Wishbone B4: WISHBONE System-on-Chip (SoC)Interconnection
Architecturefor Portable IP Cores'',
see \url{http://opencores.org/opencores,wishbone}

\bibitem{device-model}
``Linux Kernel Device Model'',
see \url{http://lxr.linux.no/#linux+v3.0.3/Documentation/driver-model/}

\bibitem{rubini}
J. Corbet, A. Rubini, G. Kroah-Hartman, ``Linux Device Drivers'', 3rd
edition. O'Reilly and Associates, Sebastopol, CA, 2005.

\bibitem{fpga-config-space} FPGA config space Open Hardware
Project, see
\url{http://www.ohwr.org/projects/fpga-config-space/}

\bibitem{comedi} See \url{http://www.comedi.org/}
\bibitem{iio} See article in \url{http://lwn.net/Articles/339674/}
\bibitem{zio-git} See \url{git://gnudd.com/zio-beta.git}

\end{thebibliography}

\end{document}
